\begin{resumo}[Abstract]
 \begin{otherlanguage*}{english}
According to the Brazilian National Institute for Educational Studies and Research (INEP), more than 90 percent of freshmen in engineering courses fail to graduate. Among the factors that influence this statistic is the gap between students and discouragement due to the large number of failures (Silva 2002). Seeking to collaborate with decreasing dropouts, this study applies theories of adaptive systems and learning objects for building an adaptive system that uses a quantization networks for nodes in an environment of interactive videos and can behave in different ways according to different profiles of students and serving as a material aid to teachers outside of the classroom. In addition, an important aspect addressed in this work is the study of the Adaptive hypermedia and in particular interactive videos, features that allow interaction between the learner and the learning material, a condition which confirms the occurrence of meaningful learning.

   \vspace{\onelineskip}
 
   \noindent 
   \textbf{Key-words}: Network Quantization by Nodes. Interactive videos. Adaptive navigation. Adaptive hypermedia. Multimedia learning.
 \end{otherlanguage*}
\end{resumo}
