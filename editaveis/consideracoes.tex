\chapter[Considerações Finais]{Considerações Finais}

Após a revisão bibliográfica, foi possível verificar que os estudos sobre Sistemas de Hipermídias Adaptativas e Objetos de Aprendizagem, em especial vídeos interativos, estão bastante avançados. Entretanto, apenas nos últimos anos houve uma aproximação entre as áreas. Além disso, a desmotivação dos professores, o desnível entre os ingressantes e o alto nível de reprovações aparecem como fatores críticos para a desistência de estudantes nos cursos relacionados a engenharia\cite{silva2005}. 

A construção de sistemas educativos adaptativos aparece como uma boa proposta para a diminuição do desnível entre estudantes e como ferramenta motivadora para alunos mais avançados, unindo recentes pesquisas sobre objetos de aprendizagem (OA) e sistemas de hipermídias adaptativas (SHA). Porém, é necessário prover mecanismos para que professores possam criar os materiais educativos nesses sistemas.

Nesse sentido, este trabalho apresentou as teorias relacionadas ao desenvolvimento de sistemas adaptativos e, em especial, a teoria de quantização de redes por nodos (QRN), que permite a construção de uma navegação global adaptativa para a rede de um curso. Essa teoria embasou o desenvolvimento do sistema, levando em consideração elementos importantes no âmbito da engenharia de \textit{software}, como arquitetura, testes, gerência de configuração e métodos ágeis. 

Sob o ponto de vista dos objetivos do trabalho, foi possível desenvolver um módulo de autoria de cursos compostos por vídeos interativos para que professores possam contribuir para a construção de uma base de cursos \textit{online} com o propósito de motivar e melhor nivelar estudantes. Foi desenvolvido também um módulo de visualização adapativa que, no momento, dá suporte apenas à ocultabilidade, mas que permite ao professor utilizar conceitos mais elaborados, como organizadores prévios, diferenciação progressiva ou ainda a reconciliação integrativa definidos por \citeonline{ausubel2000}.

No que diz respeito a navegação global adaptativa, foi possível implementar os algorítmos relacionados a quantização e testá-los unitáriamente, verificando o comportamento separado de cada equação, mas não o seu comportamento integrado, dado que durante o tempo de desenvolvimento do sistema, não se conseguiu calibrar os coeficientes relacionados a direção e blocos coesos.

Conforme esperado com o desenvolvimento desta pesquisa, surgiram vários pontos de melhoria da plataforma que não puderam ser adicionados ao desenvolvimento, devido ao escopo e tempo disponíveis. Um deles foi a verificação das teorias relacionadas à composição de vídeos interativos que incluem controle temporal e anotações no momento da construção do material, são elas as teorias dos hipervídeos \cite{Sadallah2012}.

Devido à organização da arquitetura do sistema, as modificações exigidas para a implementação de estruturas que permitam a utilização dessas teorias não seriam muito grandes e impactariam mais na lógica da visão, em componentes \textit{Polymer}, do que nas camadas do modelo ou do controle.

Ainda com a utilização das teorias de hipervídeos, outros elementos mais complexos definidos por \citeonline{ausubel2000} poderiam ser incorporados ao sistema, como os organizadores prévios, o que possibilita a diminuição do esforço do professor para a criação de cursos.

Além dos fatores relacionados às teorias de aprendizagem estudadas, existem as questões da engenharia de software aplicadas à plataforma desenvolvida. O propósito do desenvolvimento foi prover uma arquitetura estável que fosse passível de atualizações na visão sem que o modelo ou o controle fossem modificados, já que novas teorias de apresentação de conteúdo podem vir a substituir a proposta desenvolvida até o momento.

Para tal fim, existem ainda aspectos que devem ser modificados para melhorar a arquitetura e prover novos recursos, como a combinação dos componentes visuais \textit{Polymer} no momento de \textit{deploy} da aplicação (i.e. vulcanização), que pode reduzir significativamente o tamanho dos arquivos e o número de requisições ao servidor no carregamento inicial da aplicação \cite{vulcanize2013}. Essa modificação implica no ambiente de integração contínua, que poderá executar o processo de \textit{build} e teste com menos memória.

Outro ponto existente é o gerenciamento de pacotes e dependências \textit{npm}\footnote{Ferramenta de gerenciamento de pacotes Node.js. Mais informações em: \url{https://www.npmjs.com/}} para a arquitetura proposta, já que o \textit{Polymer} e algumas outras ferramentas, não são nativamente suportados pelo \textit{Meteor}, o que deve exigir a inclusão de um gerenciador de dependências \textit{npm}, como exemplo o \textit{meteor-npm} \footnote{Ferramenta de gerenciamento de pacotes \textit{npm} compatível com \textit{plugins Meteor}. Mais informações em: \url{https://www.npmjs.com/package/meteor-npm}}. Nesse mesmo ponto, ainda não existe um \textit{script} para configuração do ambiente de desenvolvimento completo, apenas o tutorial na wiki do repositório. 

Apesar do cumprimento dos objetivos e das oportunidades de evolução da plataforma desenvolvida, o aprendizado adquirido com este trabalho e a possibilidade de atuar ativamente no processo ensino-aprendizagem, contribuindo para a formação de melhores profissionais, foram os principais resultados obtidos. Espera-se com esta plataforma, que novas evoluções venham a ocorrer para aprimorar e colocar em prática as teorias estudadas e aqui aplicadas.
