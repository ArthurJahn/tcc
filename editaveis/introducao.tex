\chapter[Introdução]{Introdução}

Este capítulo apresenta a contextualização, o problema de pesquisa, a justificativa, 
os objetivos, a metodologia de pesquisa utilizada e a organização deste trabalho.

\section[Contextualização]{Contextualização}

O processo de ensino aprendizagem tem sofrido grandes mudanças desde as transformações cientificas na segunda metade do século XX. O surgimento de pesquisas sobre o aprendizado e novas formas de ensino se deu naturalmente, motivado pelo desenvolvimento do pensamento sistêmico e as novas tendências de produção industrial enxuta, que exigiam trabalhos científicos e novas propostas para o sistema educacional. Já que a forma programada e linear de ensino utilizado após a primeira revolução industrial não se adequava mais às necessidades contemporâneas de raciocínio e tomadas de decisão requeridas do profissional intelectual \cite{oliveira2009, frigotto1989}.

Por volta de 1950 aconteceram as primeiras tentativas de integrar a computação ao processo de ensino, nesse período surgiram os primeiros sistemas direcionados para a educação mediada por computador, foram os chamados Instrução Assistida por Computador, ou \textit{Computer Assisted Instruction} (CAI). Esses sistemas tinham como pressuposto teórico o Behaviorismo, em que a instrução era feita de forma linear e subdividida em diversos módulos sequenciais. Ou seja, a instrução era baseada em uma modelagem de estímulos previamente definidos \cite{giraffa1995,vicari2003}. 

Com a popularização da internet, por volta de 1990, surgiram os primeiros sistemas educativos para a web. Com base em teorias de hipermídias adaptativas \cite{brusilovsky1996} e de tutoria inteligente \cite{brusilovsky1994}, foram criados novos ambientes educacionais adaptativos e inteligentes que atuavam segundo modelos de usuário, traçados de acordo com o uso do sistema.

Nesse mesmo período apareceram as primeiras pesquisas em sistemas de vídeos interativos que, mesmo com o uso de vídeo cassete e um monitor, permitiam maior controle e interação do aprendiz com o material de aprendizado, o que corrobora para uma aprendizagem mais efetiva \cite{zhang2005}. Na época, foi possível verificar que a atividade era promissora mesmo na fase inicial de aplicação \cite{gaudreau1984}.

Já nos trabalhos mais recentes, há uma tendência para a utilização de sistemas tutores inteligentes combinados com as teorias de hipermídias adaptativas para uma melhor aprendizagem \cite{fragelli2010}. Fragelli discorre sobre uma nova abordagem para quantização de redes hipermídia que leva em consideração apenas as informações dos nodos, a Quantização de Redes por Nodos (QRN), e integra as teorias de hipermídias adaptativas, sistemas tutores inteligentes e objetos de aprendizagem (OA), que no caso específico deste trabalho, se configuram como vídeos interativos.

\section[Problema de Pesquisa]{Problema de Pesquisa}

Com base no contexto apresentado, este trabalho pretende verificar se é possível desenvolver uma navegação global adaptativa para um sistema de vídeos interativos utilizando a Quantização de Redes por Nodos. 

\section[Justificativa]{Justificativa}

Apesar dos esforços realizados no desenvolvimento de sistemas inteligentes e adaptativos no processo de ensino, poucos foram os trabalhos que se preocuparam com a implantação e manutenção prática da tecnologia proposta, o que reforça a afirmação de que as linhas de pesquisa estão mais voltadas para o campo técnico do que para o âmbito da aprendizagem. Além disso, a QRN ainda não possui validação em um ambiente real de uso, sendo avaliada por especialistas na área com análises em diferentes redes hipermídias e com diferentes perfis de usuário \cite{fragelli2010}.

Outro fator a ser analisado é que os ambientes virtuais de aprendizagem exigem um grande número de OAs para serem efetivos, já que para alcançar o público de estudantes, tem-se a necessidade de produzir materiais mais elaborados e complexos segundo as diferentes necessidades cognitivas e perfis dos usuários. Para um professor-autor, essa característica desmotiva a produção dos materiais e corrobora a para a inutilização do sistema \cite{fragelli2010}. Dessa forma, seria uma boa medida se o sistema proposto contemplasse mecanismos facilitadores para a produção de recursos educacionais.

Ademais, segundo a Sinopse de Educação Superior publicada pelo Instituto Nacional de Estudos e Pesquisas Educacionais (INEP), o índice de desistência dos cursos na área de Engenharia, Produção e Construção nas instituições públicas chega a mais de 90\% dos matriculados (figura 1). Isso indica uma grande desmotivação por parte dos alunos que pode ser explicada, dentre outros fatores, pelo alto nível de reprovações e pelo desnível entre os estudantes (SILVA 2002).
 \cite{ausubel2000}