\begin{resumo}
Segundo o Instituto Nacional de Estudos e Pesquisas Educacionais (INEP), mais de 90 por cento dos ingressantes nos cursos de engenharias não chegam a se formar. Dentre os fatores que influenciam essa estatística, está o desnível entre os estudantes e a desmotivação devido ao grande número de reprovações (SILVA, 2002). Buscando colaborar com a diminuição dos desistentes, este trabalho aplica teorias sobre sistemas adaptativos e de objetos de aprendizagem na construção de um sistema adaptativo que utiliza a Quantização de Redes por Nodos em um ambiente de vídeos interativos, que se comporta de formas distintas segundo diferentes perfis de estudantes e serve como um material de auxílio aos professores fora da sala de aula. Além disso, um aspecto importante abordado neste trabalho é o estudo das Hipermídias Adaptativas e, em especial, dos vídeos interativos, recursos que permitem a interação entre o aprendiz e o material de ensino, condição que corrobora para a ocorrência de aprendizagem significativa, confirmando a possibilidade da utilização da QRN como algoritmo para quantização da rede hipermídia.

 \vspace{\onelineskip}
    
 \noindent
 \textbf{Palavras-chaves}: Quantização de Redes por Nodos. Vídeos Interativos. Navegação Adaptativa. Hipermídias Adaptativas. Aprendizado Multimídia.
\end{resumo}
